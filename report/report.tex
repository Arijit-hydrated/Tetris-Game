\documentclass{article}

\addtolength{\oddsidemargin}{-.875in}
\addtolength{\evensidemargin}{-.875in}
\addtolength{\textwidth}{1.75in}

\addtolength{\topmargin}{-.375in}
\addtolength{\textheight}{1.75in}
\setlength{\parindent}{0pt}

\usepackage{titling}
% \usepackage{authblk}

\title{CS3243 Tetris Project}
\author{Varun Gupta \and Gao Bo \and Advay Pal \and Chang Chu-Ming \and Herbert Ilhan Tanujaya}


\date{13-04-2017}

\begin{document}

	\maketitle
	\thispagestyle{empty}
	\vspace{5mm}

    \section{Introduction}
    Tetris is likely one of the world's most famous and popular games.
    In this report, we describe how we devise an agent to play the game of Tetris.
    We use an agent that greedily picks the best possible next state from a given state,
    by using a heuristic function to approximate the value of a state. To train our heuristic
    function, we use a novel algorithm that is a combination of the well known Genetic Algorithm (GA)
    and Particle Swarm Algorithm (PSO). Our agent manages to clear AVERAGE million lines on average with a max
    of MAX million lines, demonstrating that our algorithm is effective.

    \section{Agent Strategy}
    Our agent uses a linear weighted sum of features as the heuristic function for a given state. Given a state and a piece, the agent computes
    the heuristic value for all possible next states, and then greedily picks the next state with
    the maximum heuristic value.

	\begin{center}
		$\sum w_i \times f_i(s)$
	\end{center}

    \section{Features}
    We used the following features for our heuristic function:
    \begin{itemize}
        \item Altitude Difference: The difference between the height of the highest
        column and the height of the lowest column
        \item Number of Columns With Holes: The number of columns with holes, where a hole
        is defined as an empty square directly beneath a filled square
        \item Height of the highest column
        \item Number of holes in the entire board
        \item Number of Wells: The number of columns that have a height less than that of the 2
        adjacent columns
        \item Rows cleared: The number of rows cleared for that particular move
        \item Total Column Height: The sum of the heights of all the columns
        \item Total Column Height Difference: The sum of the difference of heights between adjacent columns
        \item Column Transition: Number of adjacent squares in a column with opposite parity (where parity is defined as either full or empty)
        \item Deepest Well: Height of the lowest column
        \item Row Transition: Number of adjacent squares in a row with opposite parity (where parity is defined as either full or empty)
        \item Weighted Block: Sum of value of every square, where a square's value is equal to
		the row it is in (numbered from the bottom)
        \item Well Sum: Using the same definition of wells as above, this sums up
		the depth of every well
    \end{itemize}

    While running our training algorithms, we noticed that some features were more
	important than others. In particular, the algorithm assigned highly negative
	weights to Column Transition and Well Sum, while assigning highly positive
	weights to Number of Wells. This was slightly unexpected as we thought that
	Rows Cleared would have the largest positive weights, to predispose the algorithm
	towards clearing more rows. However, the weight for this heuristic varied wildly
	between positive and negative, indicating that perhaps sometimes the algorithm
	preferred to not greedily pick moves that were clearing rows, but rather chose
	to maintain an even surface at the top.

    \section{Our Algorithm}

    Our algorithm consists of a combination of genetic algorithm and particle swarm
	optimization (PSO) algorithm. We run both of these algorithms on populations of
	100 sets of weights. The algorithms are run on what we call ‘islands’. The idea
	is that both of the algorithms run individually, but after each generation, we
	copy the 10 best heuristics on each island to the other one.\\
	We will first describe the working principle
	of the Genetic and the PSO algorithms. Subsequently, we will
	discuss the rationale behind why we chose to juxtapose these two different
	algorithms in our search for the best set of heuristics.\\

	Genetic algorithm\\
	Each generation of our genetic algorithm consists of the following sequence of steps
	\begin{enumerate}
		\item Evaluate fitness of each set of heuristics
		\item Keep top half of population and cross-over the rest
		\item Random mutation of each feature of crossed-over heuristics with
		probability of a tenth
		\item Migration of a tenth of the population to PSO island
	\end{enumerate}
	We keep the top half of the population to ensure that each set of heuristics
	that perform well will remain within the population. The mutation introduces
	some randomness to the genetic algorithm to avoid being trapped within a local
	maximum. The probability of mutation is set at ten percent.\\

	PSO algorithm\\
	Each generation of our PSO algorithm consists of the following sequence of steps
	\begin{enumerate}
		\item Set velocity of each member to a linear weighted sum of the old velocity,
		the personal best of that member, and the global best
		\item Get new position of each member by adding its velocity to the current
		position
		\item Evaluate fitness of each set of heuristics
		\item Migration of a tenth of the population to PSO island
	\end{enumerate}

	Rationale behind combination of algorithms\\
	We ran PSO and GA individually. In doing so, we realised that PSO was in general not doing
	very well, but sometimes it made a leap and jumped by almost an order of magnitude
	in terms of lines cleared. On the other hand, Genetic seems to keep getting
	better, but only quite slowly. From these experimental results, we gathered
	that Genetic was doing more of exploitation, finding the maximum in the local area,
	while PSO was carrying out exploration, looking for good solutions all over
	the search space. Hence we thought, that perhaps if we combined the two, then
	PSO could lead Genetic to better areas, which Genetic then drills down into.

    \section{Experiments and Analysis}

    Diagram 1: Learning
    Diagram 2: Performance of agent

    Experimentation
        1. Architecture specifications on which the algorithm was implemented
        2. The time taken to train
        3. The performance of the agent

    Analysis

    \section{Scaling to Big Data}
	We parallelised our algorithm by running the PSO and Genetic islands on different cores, and
	futher by calculating the fitness of each population member on different cores.
	We then ran our algorithm on the NSCC cluster, which provides 12 cores per
	compute node.

	To demonstrate the effect of parallelisation, we compared the speed performance
	of two versions of the algorithm, single-threaded and multi-threaded version,
	sharing the same set of parameters. We started both algorithms as two separate jobs
	on the NSCC cluster at the same time, and allowed them to run for around 20
	hours before comparing the results. We used the total number of lines cleared
	in the entire time span as the baseline for comparison of the total amount of
	cpu time used. Our results showed that the single-threaded algorithm cleared
	a total of $158182993$ lines, while the multi-threaded version cleared a total
	of $1899941699$ lines. This shows that parallelisation provides about 12 times speedup,
	on the setup we used, demonstrating that we can achieve linear speedup by parallelising
	our algorithm.

	The major computational cost we were incurring was in the calculation of fitness
	values for each member of the population, as this involved running an entire
	tetris game for them. By parallelising this, we achieved significant speedup,
	reducing the computation time by an order of magnitude.



    Talk about parallelising algorithm. Mention MPI
    Get speedup

    \section{Conclusion}


    \section{References}



\end{document}
